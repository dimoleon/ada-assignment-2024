\documentclass[10pt]{article}
\usepackage{mathtools}
\usepackage{fontspec}
% \usepackage[left=1.06cm,top=0.9cm,right=1.06cm,bottom=0.49cm]{geometry}
% \setmainfont{GFS Didot}
\setmainfont{EB Garamond}


% Set page size and margins
\usepackage[a4paper,top=2cm,bottom=1.5cm,left=1cm,right=1cm]{geometry}
\usepackage{fancyhdr}
\pagestyle{fancy}
\setlength{\parindent}{0pt}

% Useful packages
\usepackage{graphicx}
\graphicspath{{./media/}}
\usepackage{subfig}
\usepackage{float}
\usepackage[colorlinks=true, allcolors=blue]{hyperref}
\usepackage{siunitx}
% \usepackage{sectsty}
% \sectionfont{\fontsize{12}{15}\selectfont}

\title{\vspace{-2cm}034 - Ανάλυση και Σχεδιασμός Αλγορίθμων\\ 
         Προαιρετική Εργασία \\
         \large Διδάσκων: Επίκ. Καθ. Παναγιώτης Πετραντωνάκης}

\author{Ομάδα 54 \\
        Ιωάννης Δημουλιός 10641 \\
        Χριστόφορος Μαρινόπουλος ΑΕΜ}
\date{Εαρινό εξάμηνο 2024}

\lhead{Προαιρετική Εργασία}
\chead{034 - Ανάλυση και Σχεδιασμός Αλγορίθμων}
\rhead{Ομάδα 54}
\begin{document}
\maketitle
\section*{Πρόβλημα 1}
\subsection*{Ερώτημα 1}
Θέλουμε να δούμε αν είναι εφικτό ένα δρομολόγιο από την πόλη \(s\) στην πόλη \(t\) δίχως να χρησιμοποιήσουμε ακμές με \(e\) απόσταση \(l_e > L\). 

Επομένως, φτιάχνουμε ένα νέο γράφο \(G' = (V, E')\), ο οποίος διαφέρει από τον αρχικό μόνο στις ακμές. Το νέο σύνολο ακμών \(E'\) δεν έχει τις ακμές που αναφέρονται παραπάνω, δηλαδή: 
\[
    E' = \{e \in E \mid l_e \leq L\}
\]

Στον γράφο \(G'\) τρέχουμε τον αλγόριθμο DFS ξεκινόντας από την κορυφή \(s\) και ελέγχουμε έτσι αν υπάρχει μονοπάτι μέχρι την κορυφή \(t\), που είναι και το ζητούμενο. 
Η χρονική πολυπλοκότητα του αλγορίθμου είναι \(O(|V| + |E|)\) αφού προσπελαύνουμε κάθε κορυφή και κάθε ακμή του γράφου \(G\) μια φορά. 



\newpage
hello world 2
\end{document}
