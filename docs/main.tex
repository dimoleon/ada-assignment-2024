\documentclass[12pt]{article}
\usepackage{mathtools}
\usepackage{amssymb}
\usepackage{fontspec}
% \usepackage[left=1.06cm,top=0.9cm,right=1.06cm,bottom=0.49cm]{geometry}
% \setmainfont{GFS Didot}
\setmainfont{EB Garamond}


% Set page size and margins
\usepackage[a4paper,top=2cm,bottom=1.5cm,left=1cm,right=1cm]{geometry}
\usepackage{fancyhdr}
\pagestyle{fancy}
\setlength{\parindent}{0pt}

% Useful packages
\usepackage{listings}
\usepackage{graphicx}
\graphicspath{{./media/}}
\usepackage{subfig}
\usepackage{float}
\usepackage[colorlinks=true, allcolors=blue]{hyperref}
\usepackage{siunitx}
% \usepackage{sectsty}
% \sectionfont{\fontsize{12}{15}\selectfont}

\title{\vspace{-2cm}034 - Ανάλυση και Σχεδιασμός Αλγορίθμων\\ 
         Προαιρετική Εργασία \\
         \large Διδάσκων: Επίκ. Καθ. Παναγιώτης Πετραντωνάκης}

\author{Ομάδα 54 \\
        Ιωάννης Δημουλιός 10641 \\
        Χριστόφορος Μαρινόπουλος ΑΕΜ}
\date{Εαρινό εξάμηνο 2024}

\lhead{Προαιρετική Εργασία}
\chead{034 - Ανάλυση και Σχεδιασμός Αλγορίθμων}
\rhead{Ομάδα 54}
\begin{document}
\maketitle

\section*{Πρόβλημα 1}
\subsection*{Ερώτημα 1}
Θέλουμε να δούμε αν είναι εφικτό ένα δρομολόγιο από την πόλη \(s\) στην πόλη \(t\) δίχως να χρησιμοποιήσουμε ακμές \(e\) με μήκος \(l_e > L\). 

Επομένως, φτιάχνουμε ένα νέο γράφο \(G' = (V, E')\), ο οποίος διαφέρει από τον αρχικό μόνο στις ακμές. Το νέο σύνολο ακμών \(E'\) δεν έχει τις ακμές που αναφέρονται παραπάνω, δηλαδή: 
\[
    E' = \{e \in E \mid l_e \leq L\}
\]

. Στον γράφο \(G'\) τρέχουμε τον αλγόριθμο DFS ξεκινόντας από την κορυφή \(s\) και ελέγχουμε έτσι αν υπάρχει μονοπάτι μέχρι την κορυφή \(t\), που είναι και το ζητούμενο. 
Η χρονική πολυπλοκότητα του αλγορίθμου είναι \(O(|V| + |E|)\), αφού προσπελαύνουμε κάθε κορυφή και κάθε ακμή του γράφου \(G\) μια φορά. \hfill \blacksquare

\subsection*{Ερώτημα 2}
Θέλουμε να βρούμε το ελάχιστο μήκος του μακρύτερου δρόμου που πρέπει να διασχίσουμε από την πόλη \(s\) στην πόλη \(t\). Αυτό το μήκος θα είναι και το ζητούμενο \(L\). 

Επομένως θα χρησιμοποιήσουμε τον αλγόριθμο Dijkstra ελαφρώς τροποποιημένο, ώστε το βάρος της κάθε κορυφής \(v\) να μην είναι πλέον το ελάχιστο άθροισμα των μηκών των ακμών ενός μονοπατιού από την κορυφή \(s\) έως την \(v\), αλλά το ελάχιστο των μέγιστων μηκών των ακμών αυτών των μονοπατιών. 

Για να γίνει πιο ξεκάθαρο, αν \(P\) το σύνολο των μονοπατιών από την κορυφή \(s\) μέχρι την \(v\) και, \(p \in P\) ένα τέτοιο μονοπάτι, τότε ορίζουμε
\[
    f(p) = \max_{e\in p} l_e
\]
και αν \(w(\cdot)\) η συνάρτηση βάρους μια κορυφής, τότε 
\[
    w(v) = \min_{p\in P} f(p)
\]

. O αλγόριθμος, λοιπόν, βρίσκει και το ζητούμενο \(w(t)\).

Το μόνο που χρειάζεται να αλλάξουμε στην υλοποίηση σε σύγκριση με τον κλασικό Dijkstra είναι η συνάρτηση ``χαλάρωσης'', ώστε να αντικατοπτριστεί η παραπάνω αλλαγή. 

Η χρονική πολυπλοκότητα του Dijkstra, εφόσον η ουρά προτεραιότητας που απαιτείται υλοποιηθεί με δυαδικό σωρό (binary heap), είναι \(O((|V| + |E|)\log |V|)\). \hfill \blacksquare 

\newpage
\section*{Πρόβλημα 2 }

\newpage
\section*{Πρόβλημα 3} 
Ορίζουμε ότι οι δείκτες πινάκων και συμβολοσειρών ξεκινούν από το \(0\) και υποθέτουμε ότι \(i+1\)-οστό στοιχείο του πίνακα \(X\) είναι το \(x_i\).
\bigskip

Δεν μας νοιάζει η συμβολοσειρά που θα χωρίσουμε αυτούσια, όσο το πλήθος των χαρακτήρων της, έστω \(n\). 

Υποθέτουμε ότι οι τομές θα γίνουν στους δείκτες \(\lambda_i\) και προσθέτουμε στον πίνακα των τομών το \(-1\) στην αρχή και το \(n-1\) στο τέλος, ώστε η αρχή και το τέλος της συμβολοσειράς να εμφανίζονται ως τομές. 

Οπότε ο πίνακας των τομών γράφεται 
\[
    B = (-1, \lambda_1, \ldots, \lambda_m, n-1)
\].
Έστω \[S = (s_0, \ldots, s_{n-1})\] η αρχική συμβολοσειρά. 

Ορίζουμε 
\[
    S_{i,j} = (s_{b_i+1},\ldots s_{b_j})
\] με \(0 \leq i < j \leq m+1\).
Tότε \(S = S_{0, m+1}\).

Με το πέρας της διαδικασίας θέλουμε να έχουμε τις \(m+1\) συμβολοσειρές 
\[
    S_{0,1}, S_{1,2}, \ldots S_{m, m+1}
\].
Ορίζουμε τον \((m+2) \times (m+2)\) πίνακα \(D\) του οποίου το στοιχείο \(d_{i,j}\) αναπαριστά τις ελάχιστες μονάδες χρόνου που απαιτούνται, ώστε με χρήση των τομών να λάβουμε από την \(S_{i, j}\) τις 
\[
    S_{i,i+1},\ldots, S_{j-1, j}\]
Σκοπός του προβλήματος, λοιπόν είναι να υπολογίσουμε το \(d_{0, m+1}\). 

Εφόσον, στην \(S_{i,i+1}\) δεν μπορούμε να κάνουμε άλλη τομή έχουμε 
\[
    d_{i,i+1} = 0 \quad \text{  για  }\quad 0 \leq i \leq m\].
Τώρα όμως για διαφορά δεικτών \(j - i > 1\) προκύπτει
\begin{align*}
    d_{i,j} &= \min_{i < k < j} \left( (d_{i,k} + b_k - b_i) + (d_{k,j} + b_j - b_k) \right) \\
            &= \min_{i < k < j} \left( d_{i,k}+ d_{k,j} + b_j - b_i \right)
\end{align*}
. Υπολογίζουμε, έτσι, με τη χρήση δυναμικού προγραμματισμού τον πίνακα \(D\) ξεκινώντας από την κύρια διαγώνιό του, προσπελαύνοντας τις διαγωνίους σταθερής διαφοράς δεικτών, έως ότου φτάσουμε στο στοιχείο \(d_{0, m+1}\). 

\bigskip
Η χρονική πολυπλοκότητα του αλγορίθμου προκύπτει ως εξής: 
\begin{itemize}
    \item Προσπελαύνουμε τα μισά (όλα πάνω από την κύρια διαγώνιο) στοιχεία του πίνακα D διαστάσεων \((m+2)\times (m+2)\), άρα \(O(m^2)\).
    \item Για τον υπολογισμό κάθε \(d_{i,j}\) έχουμε έναν βρόχο με το πολύ \(m\) επαναλήψεις, άρα \(O(m)\).
\end{itemize}
Επομένως, τελικά είναι \(O(m^3)\). \hfill \blacksquare
\end{document}
